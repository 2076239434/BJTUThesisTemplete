%========================================================================%
%               Copyright (C) 2016 All Rights Reserved.                  %
%             Author:BillHu<billhu@icloud.com>  Ver:<1.0>                %
%========================================================================%
%    The author grants permission, without fee and without a written     %
% license agreement, for use, reproduction, modification, and distribu-  %
% tion of this software and its documentation by educational, research,  %
% and non-profit entities for noncommercial purposes only.The above      %
% copyright notice and this paragraph MUST appear in all copies and      %
% modifications of the software and/or documentation.                    %
%========================================================================%
\documentclass{bjtuthesis}
%========================================================================%
% 自定义内容
%========================================================================%
\ctitle{中文题目}
\etitle{English Title}
\cauthor{作者}
\ctutor{导师}
\category{工学}
\major{土木工程}
\cid{14121023}
\jobtitle{教授}
\degree{硕士}
\field{桥梁工程}
\cdata{2017年3月}
%========================================================================%
% 自定义文字
%========================================================================%
\cthanks{放置在摘要页前,对象包括:1)国家科学基金,资助研究工作的奖学金基金,合同单位,资助或支持的企业、组织或个人。2)协助完成研究工作和提供便利条件的组织或个人。3)在研究工作中提出建议和提供帮助的人。4)给予转载和引用权的资料、图片、文献、研究思想和设想的所有者。5)其他应感谢的组织和个人。}
\cabstract{中文摘要应将学位论文的内容要点简短明了地表达出来,硕士学位论文一般为500~1000字,博士学位论文一般为1000~2000字。留学生英文版学位论文不少于3000字中文摘要,留学生英文版博士学位论文不少于5000字中文摘要。字体为宋体小四号。内容应包括工作目的、研究方法、成果和结论。要突出本论文的创新点,语言力求精炼。为了便于文献检索,应在本页下方另起一行注明论文的关键词(3-8个),如有可能,尽量采用《汉语主题词表》等词表提供的规范词。图X幅,表X个,参考文献X篇。

中文摘要应将学位论文的内容要点简短明了地表达出来,硕士学位论文一般为500~1000字,博士学位论文一般为1000~2000字。留学生英文版学位论文不少于3000字中文摘要,留学生英文版博士学位论文不少于5000字中文摘要。字体为宋体小四号。内容应包括工作目的、研究方法、成果和结论。要突出本论文的创新点,语言力求精炼。为了便于文献检索,应在本页下方另起一行注明论文的关键词(3-8个),如有可能,尽量采用《汉语主题词表》等词表提供的规范词。图X幅,表X个,参考文献X篇。

中文摘要应将学位论文的内容要点简短明了地表达出来,硕士学位论文一般为500~1000字,博士学位论文一般为1000~2000字。留学生英文版学位论文不少于3000字中文摘要,留学生英文版博士学位论文不少于5000字中文摘要。字体为宋体小四号。内容应包括工作目的、研究方法、成果和结论。要突出本论文的创新点,语言力求精炼。为了便于文献检索,应在本页下方另起一行注明论文的关键词(3-8个),如有可能,尽量采用《汉语主题词表》等词表提供的规范词。图X幅,表X个,参考文献X篇。

中文摘要应将学位论文的内容要点简短明了地表达出来,硕士学位论文一般为500~1000字,博士学位论文一般为1000~2000字。留学生英文版学位论文不少于3000字中文摘要,留学生英文版博士学位论文不少于5000字中文摘要。字体为宋体小四号。内容应包括工作目的、研究方法、成果和结论。要突出本论文的创新点,语言力求精炼。为了便于文献检索,应在本页下方另起一行注明论文的关键词(3-8个),如有可能,尽量采用《汉语主题词表》等词表提供的规范词。图X幅,表X个,参考文献X篇。

中文摘要应将学位论文的内容要点简短明了地表达出来,硕士学位论文一般为500~1000字,博士学位论文一般为1000~2000字。留学生英文版学位论文不少于3000字中文摘要,留学生英文版博士学位论文不少于5000字中文摘要。字体为宋体小四号。内容应包括工作目的、研究方法、成果和结论。要突出本论文的创新点,语言力求精炼。为了便于文献检索,应在本页下方另起一行注明论文的关键词(3-8个),如有可能,尽量采用《汉语主题词表》等词表提供的规范词。图X幅,表X个,参考文献X篇。
中文摘要应将学位论文的内容要点简短明了地表达出来,硕士学位论文一般为500~1000字,博士学位论文一般为1000~2000字。留学生英文版学位论文不少于3000字中文摘要,留学生英文版博士学位论文不少于5000字中文摘要。字体为宋体小四号。内容应包括工作目的、研究方法、成果和结论。要突出本论文的创新点,语言力求精炼。为了便于文献检索,应在本页下方另起一行注明论文的关键词(3-8个),如有可能,尽量采用《汉语主题词表》等词表提供的规范词。图X幅,表X个,参考文献X篇。
}
\eabstract{Something Something Something Something Something Something Something Something Something Something Something Something Something Something Something Something Something Something Something Something.}
\cprelude{学位论文的序或前言,一般是作者或他人对本篇论文基本特征的简介,如说明研究工作缘起、背景、主旨、目的、意义、编写体例,以及资助、支持、协作经过等;也可以评述和对相关问题发表意见。这些内容也可以在正文引言中说明。}
%========================================================================%
% 前置部分
%========================================================================%
\begin{document}
\pagenumbering{roman}
\cover
\myclpage
\ccopyright
\myclpage
\setcounter{page}{1}
\ctitlepage
\myclpage
\thankspage
\myclpage
\cabspage
\myfanpage
\eabspage
\myfanpage
\cprepage
\myfanpage
\pagestyle{myfancy}
\tableofcontents
%========================================================================%
% 主体部份
%========================================================================%
\chapter{绪论}
\markboth{绪论}{}
\pagenumbering{arabic}
\setcounter{page}{1}
我是绪论。我是绪论。我是绪论。我是绪论。我是绪论。我是绪论。我是绪论。我是绪论。
\chapter{韩冉}
\chapter{钢筋混凝土曲线桥的最不利激励角研究}
\section{引言}
A首先,阐述现状。一般研究、设计都是计算顺桥向和(或)横桥向激励(顺、横);对于曲线桥,往往研究
(割线、垂割、切)

B抛出问题:对么?


C研究的问题是:

1、某响应({\bf 大小、方向、时刻})是否存在与向量角度不一致的最不利激励角?
\begin{equation}
response(loc_r,mag_r,dir_r,t)=input(loc_i,mag_i,dir_i,t)
\end{equation}

2、如果存在,如何确定这个激励角?


比如,墩底顺桥向弯矩$M$,响应的方向是“顺桥向”,那么使得该响应最大的激励角是不是顺桥向?如果不是,是什么方向?

\section{激励-响应角的不一致性}
\subsection{最不利角度差的定义}
用角度差衡量不一致性

角度差为最不利输入方向与改点切线方向的角度之差,并规定向曲线外侧为正,如图所示。




\subsection{标准参数化模型}
为了定性的研究本问题,这里建立了曲线桥的标准模型,如图所示。
曲率、墩高、边中跨比、跨高比
\subsection{标准激励}

\subsection{结果验证}

\subsubsection{横桥向墩底弯矩}
顺:1
横:10
切:3

但,当激励角与切线夹角为20度时,响应为20

\subsubsection{顺桥向桥台加速度}

顺:10
横:1
切:3

但,当激励角与切线夹角为10度时,响应为20

从以上分析可见,无论是位移响应还是内力响应,无论是基础结构还是上部结构,都存在激励响应角不一致的问题。本章将详细讨论两者的关系。

\section{最不利角度差与曲率的关系}
\section{最不利角度差与边中跨比的关系}
\section{最不利角度差与跨高比的关系}
\section{精华}
\section{结果验证}

\subsubsection{横桥向墩底弯矩}
\subsubsection{顺桥向桥台加速度}

\section{本章小结}

%========================================================================%
% 参考文献
%========================================================================%
\bibliography{MyDissBib}
\addcontentsline{toc}{part}{参考文献}
\cleardoublepage 
%========================================================================%
% 附录
%========================================================================%
\markboth{附录A}{}
\addcontentsline{toc}{part}{附录A}
\begin{center}
{\zihao{3}\heiti 附录A}
\end{center}
\begin{center}
{\zihao{3}\heiti 附录标题}
\end{center}

\indent\zihao{5}附录是作为论文主体的补充项目,并不是必须的。
论文的附录依序用大写正体英文字母A、B、C……编序号,如:附录A。
\cleardoublepage
%========================================================================%
% 索引
%========================================================================%
\markboth{索引}{}
\addcontentsline{toc}{part}{索引}
\begin{center}
{\zihao{3}\heiti 索引}
\end{center}

\indent\zihao{5}按照需要编排分类索引、著者索引、关键词索引等。
\cleardoublepage
%========================================================================%
% 作者简历及研究成果
%========================================================================%
\markboth{作者简历及攻读硕士学位期间取得的研究成果}{}
\addcontentsline{toc}{part}{作者简历及攻读硕士学位期间取得的研究成果}
\begin{thecenter}
作者简历及攻读硕士学位期间取得的研究成果
\end{thecenter}

\zihao{5}包括教育经历、工作经历、攻读学位期间发表的论文和完成的工作等。行距16磅,段前后各为0磅。
\cleardoublepage
\cstatement
\clastpage
\end{document}
